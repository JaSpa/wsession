\documentclass[acmsmall,screen,anonymous,review]{acmart}

\usepackage{agda}

%% There is a limit of 25 pages for a full paper or Functional Pearl
%% and 12 pages for an Experience Report; in either case, the
%% bibliography and an optional clearly marked appendix will not be
%% counted against these limits. Submissions that exceed the page
%% limits or, for other reasons, do not meet the requirements for
%% formatting, will be summarily rejected. 


%% Rights management information.  This information is sent to you
%% when you complete the rights form.  These commands have SAMPLE
%% values in them; it is your responsibility as an author to replace
%% the commands and values with those provided to you when you
%% complete the rights form.
\setcopyright{acmcopyright}
\copyrightyear{2023}
\acmYear{2023}
\acmDOI{XXXXXXX.XXXXXXX}


%%
%% These commands are for a JOURNAL article.
% \acmJournal{JACM}
% \acmVolume{37}
% \acmNumber{4}
% \acmArticle{111}
% \acmMonth{8}

%%
%% Submission ID.
%% Use this when submitting an article to a sponsored event. You'll
%% receive a unique submission ID from the organizers
%% of the event, and this ID should be used as the parameter to this command.
%%\acmSubmissionID{123-A56-BU3}

%%
%% For managing citations, it is recommended to use bibliography
%% files in BibTeX format.
%%
%% You can then either use BibTeX with the ACM-Reference-Format style,
%% or BibLaTeX with the acmnumeric or acmauthoryear sytles, that include
%% support for advanced citation of software artefact from the
%% biblatex-software package, also separately available on CTAN.
%%
%% Look at the sample-*-biblatex.tex files for templates showcasing
%% the biblatex styles.
%%

%%
%% The majority of ACM publications use numbered citations and
%% references.  The command \citestyle{authoryear} switches to the
%% "author year" style.
%%
%% If you are preparing content for an event
%% sponsored by ACM SIGGRAPH, you must use the "author year" style of
%% citations and references.
%% Uncommenting
%% the next command will enable that style.
%%\citestyle{acmauthoryear}


%%
%% end of the preamble, start of the body of the document source.
\usepackage{dsfont}
\usepackage{newunicodechar}
\newunicodechar{λ}{\ensuremath{\mathnormal\lambda}}
\newunicodechar{σ}{\ensuremath{\mathnormal\sigma}}
\newunicodechar{τ}{\ensuremath{\mathnormal\tau}}
\newunicodechar{π}{\ensuremath{\mathnormal\pi}}
\newunicodechar{ℕ}{\ensuremath{\mathbb{N}}}
\newunicodechar{∷}{\ensuremath{::}}
\newunicodechar{≡}{\ensuremath{\equiv}}
\newunicodechar{∀}{\ensuremath{\forall}}
\newunicodechar{ᴸ}{\ensuremath{^L}}
\newunicodechar{ᴿ}{\ensuremath{^R}}
\newunicodechar{ʳ}{\ensuremath{^r}}
\newunicodechar{ⱽ}{\ensuremath{^V}}
\newunicodechar{⟧}{\ensuremath{\rrbracket}}
\newunicodechar{⟦}{\ensuremath{\llbracket}}
\newunicodechar{⊤}{\ensuremath{\top}}
\newunicodechar{⊥}{\ensuremath{\bot}}
\newunicodechar{₁}{\ensuremath{_1}}
\newunicodechar{₂}{\ensuremath{_2}}
\newunicodechar{∈}{\ensuremath{\in}}
\newunicodechar{₀}{\ensuremath{_0}}
\newunicodechar{′}{\ensuremath{'}}
\newunicodechar{ˢ}{\ensuremath{^S}}
\newunicodechar{ᴬ}{\ensuremath{^A}}
\newunicodechar{∘}{\ensuremath{\circ}}
\newunicodechar{𝟙}{\ensuremath{\mathds{1}}}  
\newunicodechar{𝟘}{\ensuremath{\mathds{O}}}
% \newunicodechar{𝟙}{\ensuremath{\mathbb{I}}}  
% \newunicodechar{𝟘}{\ensuremath{\mathbb{O}}}
\newunicodechar{ᴾ}{\ensuremath{^P}}
\newunicodechar{ᵀ}{\ensuremath{^T}}
\newunicodechar{⊎}{\ensuremath{\uplus}}
\newunicodechar{ι}{\ensuremath{\iota}}
\newunicodechar{⇐}{\ensuremath{\Leftarrow}}
\newunicodechar{⇒}{\ensuremath{\Rightarrow}}
\newunicodechar{∎}{\ensuremath{\mathnormal\blacksquare}}
\newunicodechar{➙}{\ensuremath{\to^P}}
\newunicodechar{Δ}{\ensuremath{\Delta}}
\newunicodechar{∅}{\ensuremath{\emptyset}}
\newunicodechar{⁺}{\ensuremath{^+}}
\newunicodechar{𝕏}{\ensuremath{\mathbb{X}}}
\newunicodechar{∙}{\ensuremath{\cdot}}
\newunicodechar{⁇}{\ensuremath{?}}
\newunicodechar{‼}{\ensuremath{!}}
\newunicodechar{⊕}{\ensuremath{\oplus}}
\newunicodechar{ℤ}{\ensuremath{\mathbb{Z}}}
\newunicodechar{μ}{\ensuremath{\mu}}
\newunicodechar{∃}{\ensuremath{\exists}}
\newunicodechar{⨟}{\ensuremath{\fatsemi}}
\newunicodechar{Σ}{\ensuremath{\Sigma}}
\newunicodechar{ᵣ}{\ensuremath{_r}}

\newcommand\Asuc{\AgdaInductiveConstructor{suc}}
\newcommand\Azero{\AgdaInductiveConstructor{zero}}
\newcommand\Anat{\AgdaInductiveConstructor{nat}}
\newcommand\Aint{\AgdaInductiveConstructor{int}}
\newcommand\Abool{\AgdaInductiveConstructor{bool}}
\newcommand\Atend{\AgdaInductiveConstructor{end}}
\newcommand\Atsend[2]{\AgdaInductiveConstructor{‼{\textcolor{black}{\ensuremath{#1}}}∙{\textcolor{black}{\ensuremath{#2}}}}}
\newcommand\Atrecv[2]{\AgdaInductiveConstructor{⁇{\textcolor{black}{\ensuremath{#1}}}∙{\textcolor{black}{\ensuremath{#2}}}}}
\newcommand\ACEND{\AgdaInductiveConstructor{END}}
\newcommand\ACSEND{\AgdaInductiveConstructor{SEND}}
\newcommand\ACRECV{\AgdaInductiveConstructor{RECV}}
\newcommand\ACSELECT{\AgdaInductiveConstructor{SELECT}}
\newcommand\ACCHOICE{\AgdaInductiveConstructor{CHOICE}}
\newcommand\AFin{\AgdaDatatype{Fin}}
\newcommand\ACommand{\AgdaDatatype{Command}}
\newcommand\ACommandStore{\AgdaDatatype{CommandStore}}
\newcommand\ASession{\AgdaDatatype{Session}}
\newcommand\Abinaryp{\AgdaFunction{binaryp}}
\newcommand\Aunaryp{\AgdaFunction{unaryp}}
\newcommand\Aexecutor{\AgdaFunction{exec}}
\newcommand\Aexec{\AgdaFunction{exec}}
\newcommand\AIO{\AgdaFunction{IO}}
\newcommand\Amu{\AgdaInductiveConstructor{\ensuremath{\mu}}}
\newcommand\AMU{\AgdaInductiveConstructor{LOOP}}
\newcommand\ACONTINUE{\AgdaInductiveConstructor{CONTINUE}}
\newcommand\Aback{\AgdaInductiveConstructor{\ensuremath{`}}}
\newcommand\Amanyunaryp{\AgdaFunction{many-unaryp}}
\newcommand\Arestart{\AgdaFunction{restart}}

%%% Local Variables:
%%% mode: latex
%%% TeX-master: "main-icfp23-pearl"
%%% End:

\begin{document}

%%
%% The "title" command has an optional parameter,
%% allowing the author to define a "short title" to be used in page headers.
\title{How to Embed Session Types Without Linearity}

%%
%% The "author" command and its associated commands are used to define
%% the authors and their affiliations.
%% Of note is the shared affiliation of the first two authors, and the
%% "authornote" and "authornotemark" commands
%% used to denote shared contribution to the research.
\author{Peter Thiemann}
% \authornote{Both authors contributed equally to this research.}
\email{thiemann@acm.org}
\orcid{0000-0002-9000-1239}
% \author{G.K.M. Tobin}
% \authornotemark[1]
% \email{webmaster@marysville-ohio.com}
\affiliation{%
  \institution{University of Freiburg}
  % \streetaddress{P.O. Box 1212}
  % \city{Dublin}
  % \state{Ohio}
  \country{Germany}
  % \postcode{43017-6221}
}

%%
%% By default, the full list of authors will be used in the page
%% headers. Often, this list is too long, and will overlap
%% other information printed in the page headers. This command allows
%% the author to define a more concise list
%% of authors' names for this purpose.
% \renewcommand{\shortauthors}{Trovato et al.}

%%
%% The abstract is a short summary of the work to be presented in the
%% article.
\begin{abstract}
  TBD
\end{abstract}

%%
%% The code below is generated by the tool at http://dl.acm.org/ccs.cfm.
%% Please copy and paste the code instead of the example below.
%%
% \begin{CCSXML}
% <ccs2012>
%  <concept>
%   <concept_id>10010520.10010553.10010562</concept_id>
%   <concept_desc>Computer systems organization~Embedded systems</concept_desc>
%   <concept_significance>500</concept_significance>
%  </concept>
%  <concept>
%   <concept_id>10010520.10010575.10010755</concept_id>
%   <concept_desc>Computer systems organization~Redundancy</concept_desc>
%   <concept_significance>300</concept_significance>
%  </concept>
%  <concept>
%   <concept_id>10010520.10010553.10010554</concept_id>
%   <concept_desc>Computer systems organization~Robotics</concept_desc>
%   <concept_significance>100</concept_significance>
%  </concept>
%  <concept>
%   <concept_id>10003033.10003083.10003095</concept_id>
%   <concept_desc>Networks~Network reliability</concept_desc>
%   <concept_significance>100</concept_significance>
%  </concept>
% </ccs2012>
% \end{CCSXML}

% \ccsdesc[500]{Computer systems organization~Embedded systems}
% \ccsdesc[300]{Computer systems organization~Redundancy}
% \ccsdesc{Computer systems organization~Robotics}
% \ccsdesc[100]{Networks~Network reliability}

%%
%% Keywords. The author(s) should pick words that accurately describe
%% the work being presented. Separate the keywords with commas.
\keywords{session types, domain specific languages, linear types}

% \received{20 February 2007}
% \received[revised]{12 March 2009}
% \received[accepted]{5 June 2009}

%%
%% This command processes the author and affiliation and title
%% information and builds the first part of the formatted document.
\maketitle

\section{Introduction}
\label{sec:introduction}

Session types provide a type discipline for structured communication
in concurrent programming systems.

We concentrate on binary session types, although our proposal should
be extensible to the multi-party setting.

For a long time, session types were inextricably connected to linear
types. This connection has affected their implementation and widespread
adoption. Languages constructed around session types embrace linearity
and their type checker rejects violations thereof. Examples: Links,
Sepi, Sill, C0, more?
While these languages and their implementation have fostered research
and encouraged experimentation, these languages are not widely used.

To boost the use of session types, 
a lot of work is dedicated to embedding session types in mainstream
languages, most of which do not have native support for linearity. 
(Plenty of examples with different trade-offs.)

Recent work on multi-party session types suggests an alternative
approach, reminiscent of the inversion of control approach which is
familiar to programmer from GUI programming. The system described in
that paper compiles a description of a multi-party session type into
a library that encapsulates the implementation of all
communication. For each communication action, the library provides an
interface where the programmer specifies a callback function for this
particular action.

We show how to apply this idea to binary session types in the context
of dependently-typed functional programming. Our vehicle is the
language Agda, so that our development is transferrable to Haskell,
either via compilation or via translation
\cite{DBLP:conf/haskell/CockxME0N22}.

\section{Finite nonbranching simple session types}
\label{sec:finite-nonbr-simple}

\input{latex/ST-finite-nonbranching.tex}

Let's start straight away with the simplest instance, finite
nonbranching simple session types, to convey the
gist of the approach. Subsequent sections show how to add most of the
usual features of session types.

A binary session type describes a bidirectional communication between two peers,
let's call them server and client. The session type is attached to the
type of the communication channel.
\stFiniteType
\stFiniteSession
These types correspond to the standard grammar of session types, where
$T$ is the type of payloads that can be transmitted and $S$ is the
type of sessions. 
\begin{align*}
  T &::= \Anat & S &::= \Atsend{T}{S} \mid \Atrecv{T}{S} \mid \Atend
\end{align*}
The session type $\Atsend{T}{S}$ ($\Atrecv{T}{S}$) describes a channel that is ready to send (receive)
a value of payload type $T$ and the continue as $S$. The session type
$\Atend$ describes a channel that can only be closed.

Here are two examples for session types: the types of the server for a
binary operation and a unary operation, respectively.
\stExampleBinpUnP

In standard session type theory \cite{???}, there are primitives to
send and receive values and to close a channel with types like this:
\begin{align*}
  \mathtt{send} & : \Atsend{T}{S} \otimes T \multimap S &
                                                   \mathtt{recv} &:
                                                                   \Atrecv{T}{S}
                                                                   \multimap
                                                                   (T
                                                                   \otimes
                                                                   S)
  & \mathtt{close} &: \Atend \multimap ()
\end{align*}
The types indicate that we must treat channel values of type $S$
\emph{linearly}: the \texttt{send} operation \emph{consumes} a
channel, which is ready to send, and returns it in a state described
by $S$; the \texttt{recv} operation \emph{consumes} a channel, which
is ready to receive, and returns a pair with the received value and
the updated channel; the \texttt{close} operation \emph{consumes} the
channel and returns a unit value. Enforcing this linearity is required
for soundness.

We take a different approach inspired by programming with
callbacks. Instead of providing \texttt{send} and \texttt{recv}
primitives to the programmer, we ask the programmer to define the
``application logic'' by implementing a value whose type is indexed by
a session type.
\stCommand
In this definition, the type $A$ embodies the application state. 
Each $\ACSEND$ command takes a state transformer that extracts the value to
send from the current application state; each $\ACRECV$ command takes
a state transformer that is indexed by the received value; the $\ACEND$
command terminates the session. In fact, we could provide the
application logic by actions in a state monad over the application
state $A$. We defer the shift to a monadic interface to the next step,
when we have the full picture.

Continuing our example, we define commands that implement a server
for the protocols {\Abinaryp} and {\Aunaryp} with the operation instantiated to
addition and negation, respectively.
\stAddpCommand
We use the variable $a$ for the application state and $x$ and $y$ for
the value received from the channel.

Finally, we need to execute commands. To this end, we write an
interpreter for commands that relies on primitive operations provided
in the $\AIO$ monad. The point of our approach is that this
interpreter is the single definition where we are obliged to prove
that it handles channels in a linear fashion. 
\stPostulates
This API should be self-explanatory. It declares an abstract type of
channels with operations to accept a connection, close a channel, as
well as send and receive a value over the channel. It glosses over issues like
serialization, which can be addressed using type classes.
% Our shared memory implementation relies on unsafe casts.

The interpreter itself is defined by induction on the type
{\ACommand}.
\stExecutorSignature\vspace{-1.5\baselineskip}
\stExecutor
To actually run a server, it remains to provide a wrapper that accepts
a connection and invokes the interpreter.
\stAcceptor
Examining the interpreter, we finally see the full monadic
structure. We need stack of monad transformers starting with a state
monad for the application state on top a reader monad providing the
channel on top of the IO monad. 

\section{Selection and choice}
\label{sec:select-choice}

Adding branching to our development is straightforward. The standard
theory allows branching on a finite set of labels using this syntax:
\begin{align*}
  S & ::= \dots \mid \oplus\{ \ell : S_\ell \mid \ell \in L \} \mid
      \&\{\ell: S_\ell \mid \ell \in L\}
\end{align*}
Here $L$ is a finite set of labels, which can be chosen differently at
every instance. 
The first alternative corresponds to an \emph{internal choice} of the
program.  The \texttt{select} primitive sends one of the labels, say $\ell \in L$, available in the
type and continues according to $S_\ell$:
\begin{align*}
  \mathtt{select}\ \ell &: \oplus\{ \ell : S_\ell \mid \ell \in L \}
                          \multimap S_\ell
\end{align*}
The second alternative corresponds to an \emph{external choice}. The
primitive \texttt{branch} receives one of the labels 
mentioned in the types and chooses a continuation according to the
label. In the presence of sum types, the primitive can be typed as
follows \cite{DBLP:journals/toplas/Padovani19}.
\begin{align*}
  \mathtt{branch} &: \&\{\ell: S_\ell \mid \ell \in L\} \multimap +\{\ell: S_\ell \mid \ell \in L\}
\end{align*}
The modeling in Agda extends the definitions of
{\ASession}, {\ACommand}, and {\Aexecutor} from
Section~\ref{sec:finite-nonbr-simple}. A label set of size $n$ is
modeled by the type $\AFin~n$ and the alternative continuation sessions by
functions from labels to {\ASession} (isomorphic to vectors of sessions, cf.\  Section~\ref{sec:select-choice-with}).
\stBranchingType
\stBranchingCommand
\stExecutorSignature\vspace{-1.5\baselineskip}
\stBranchingExecutor

Continuing our example, we define the type of an arithmetic server,
which gives a choice between a binary operation and a unary one. 
This definition uses a smart constructor for the external choice that
takes a vector and transforms it into the corresponding function.
\stExampleArithP
The command for the server extends in the obvious way. The vector
trick does not work in this case because the branches of the function
have different types.
\stArithpCommand

\section{Going monadic}
\label{sec:going-monadic}

We already remarked about the monadic structure apparent in the
callbacks and in the implementation of the {\Aexecutor}
function. Indeed, moving on to a monadic interface makes our session
programs more concise.



\newpage

\section{Discussion}
\label{sec:discussion}

\subsection{Command fragments}
\label{sec:command-fragments}


Instead of having commands indexed by session types, we could have
\emph{command fragments} indexed by transformations of session types
($S \to S$). Everything is definable in this way, but defining
protocols is awkward as the type checker is unable to provide
guidance.

Here is an example: (elaborate)

\subsection{Selection and choice with vectors}
\label{sec:select-choice-with}

The reader may wonder why we do not define the constructors for
selection and choice using vectors of length $n$, rather than
functions from $\AFin~n$ (which is isomorphic). Using function extends
straightforwardly to the definition of {\ACommand} where the {\ASession} index of the
continuation command depends on the function argument $i : \AFin~n$.
We would have to define a special vectors type to achieve similar
expressivity. 

%%
%% The acknowledgments section is defined using the "acks" environment
%% (and NOT an unnumbered section). This ensures the proper
%% identification of the section in the article metadata, and the
%% consistent spelling of the heading.
% \begin{acks}
% To Robert, for the bagels and explaining CMYK and color spaces.
% \end{acks}

%%
%% The next two lines define the bibliography style to be used, and
%% the bibliography file.
\bibliographystyle{ACM-Reference-Format}
\bibliography{references}


%%
%% If your work has an appendix, this is the place to put it.
% \appendix


\end{document}
\endinput
%%
%% End of file `sample-acmsmall-submission.tex'.
